\section{Research Experience}

\cventry{2024--Present}{Explainable AI for First-Person Video Segmentation in Nursing Simulations}{Collaborators: Daniel Levin, Gautam Biswas, Alyssa White}{}{}{
\begin{itemize}
  \item Developing explainable AI methods to analyze video segments from Tobii Glasses' first-person perspective during nursing simulation training sessions
  \item Designed an unsupervised segmentation method optimized for processing long videos efficiently
  \item Focused on interpretable models to link gaze dynamics with task performance and learning outcomes
\end{itemize}
}

\cventry{2024--Present}{IMU-Guided Segmentation and Sampling for Video Classification}{Collaborators: Meiyi Ma}{}{}{
\begin{itemize}
  \item Creating an IMU-guided method to enhance classification accuracy and efficiency in multimodal data
  \item Improved Temporal Segment Networks by incorporating motion-based insights for better frame selection
\end{itemize}
}

\cventry{2024--2024}{Continual Multitask Learning}{Collaborators: Meiyi Ma}{}{}{
\begin{itemize}
  \item Developed a Continual Multitask Learning framework, addressing challenges in continual multitask learning without requiring replay buffers
\end{itemize}
}

\cventry{2023--2024}{Star-based Reachability Verification for Targeted and Robust XAI}{Collaborators: Meiyi Ma, Taylor Johnson, Diego Manzanas Lopez}{}{}{
\begin{itemize}
  \item Led development of framework to evaluate comprehensiveness of attribution methods
  \item Utilized Neural Network Verification (NNV) to analyze boundaries of sampling-based attribution methods
  \item Designed experiments showcasing method's robustness in providing deterministic explainability
\end{itemize}
}

\cventry{2023--2024}{EXACT: A Meta-Learning Framework for Precise Exercise Segmentation in Physical Therapy}{Collaborators: Meiyi Ma}{}{}{
\begin{itemize}
  \item Led development of EXACT, a novel method for segmenting exercises within multivariate time series data using PyTorch
  \item Designed U-Net architecture with temporal positional encoding for exercise phase identification
  \item Conducted extensive experiments demonstrating superiority over traditional segmentation techniques
  \item Developed modular Python framework for easy replication and experimentation
\end{itemize}
}

\cventry{2022--2023}{MicroXercise: A Micro-Level Comparative and Explainable System for Remote Physical Therapy}{Collaborators: Meiyi Ma, Pamela Wisniewski}{}{}{
\begin{itemize}
  \item Led development of MicroXercise integrating Siamese Neural Networks with saliency maps
  \item Designed Siamese Neural Network for similarity determination and attribution scoring
  \item Incorporated saliency map techniques for explainability across modalities
  \item Conducted mixed-methods study with interviews, surveys, and quantitative analysis
\end{itemize}
}

\cventry{2021--2022}{PhysiQ: Off-Site Quality Assessment of Exercises in Physical Therapy}{Collaborators: Meiyi Ma}{}{}{
\begin{itemize}
  \item Led development of PhysiQ framework for continuous tracking of off-site exercise activity
  \item Designed multi-task spatiotemporal Siamese Neural Network for quality assessment
  \item Collected and annotated data for 31 participants with varying exercise quality levels
  \item Achieved 89.67% detection accuracy and 0.949 R-squared correlation in similarity comparison
\end{itemize}
}


\section{Publications}

\cvitem{2024}{
\textbf{Wang, Hanchen David}, Bae, Siwoo, Sun, Xutong, Thatigotla, Yashvitha, Ma, Meiyi. 
\textit{EXACT: A Meta-Learning Framework for Precise Exercise Segmentation in Physical Therapy}.
International Conference on Cyber-Physical Systems (ICCPS), November 2024, Under Review.
}

\cvitem{2024}{
\textbf{Wang, Hanchen David}, Bae, Siwoo, Chen, Zirong, Ma, Meiyi. 
\textit{Learning with Preserving for Continual Multitask Learning}.
International Conference on Learning Representations (ICLR), October 2024, Under Review.
}

\cvitem{2024}{
Cohn, Clayton, Davalos, Eduardo, Vatral, Caleb, Fonteles, Joyce, \textbf{Wang, Hanchen David}, Ma, Meiyi, Biswas, Gautam. 
\textit{Multimodal Methods for Analyzing Learning and Training Environments: A Systematic Literature Review}.
ACM Computing Surveys, August 2024, Under Review.
}


\section{Skills}

\cvitem{Programming}{Python, Java, C/C++, JavaScript, TypeScript, Swift}
\cvitem{Frameworks}{PyTorch, TensorFlow, Angular, Node.js, React}
\cvitem{Tools}{Git, Docker, LaTeX, Unity}
\cvitem{Domains}{Machine Learning, Deep Learning, Computer Vision, Healthcare AI, Explainable AI}
